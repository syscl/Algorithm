\documentclass[10pt]{article}
\usepackage{amsmath, mathrsfs}  % The amsmath package is included for \xrightarrow
\usepackage{amssymb} % for mathematic symbol 
\usepackage{amsthm} % for the proof square at the end when we finish the proof
\usepackage{listings}
\usepackage[colorlinks]{hyperref}
\usepackage{titlesec}
\usepackage{color} % for text color % \color{blue}
\usepackage{booktabs} % for excel
\usepackage{array} % for excel
\usepackage{multirow} % for excel
\usepackage{float} % solve pictures and sheets not follow by words! This issue is just the same as the fucking macOS/iOS' Pages
% citation style
\bibliographystyle{ieeetr}

\titleformat*{\section}{\normalsize\bfseries}  % titile format

\usepackage{graphicx} % insert pics

\hypersetup{allcolors=black} % cite and refer color 

\begin{document}
	\title{%
		\textbf{Correctness of the Fisher Yate Shuffle}} 
		\author{
			Author: syscl/Yating Zhou\\ 
		}
	\maketitle
		The code is attached in the same repository as \textbf{code.cc}. And we have to prove the correctness of the algorithm.\\
		\textbf{Theorem} The probability of each element in each position from the returned array is the same. 
		\begin{proof}
			For the base case, if the returned array has size 1, the probability of 1 stays in 1 is 1, and if the returned array has size 2, the probability of 1 stays in 1 is when we do not swap 1 and 2, so that is $\frac{1}{2}$, and the probability of 1 stays in the second position is thus $\frac{1}{2}$. For the inductive steps, suppose the probability of each element in each position of an array with size $k$ is $\frac{1}{k}$, then for a given element to stay in a position $1\leq i\leq k$ is thus when we do not swap this element at the $i$-th position (i.e., we swap elements other than $i$-th element), that is the probability of $\frac{1}{k}(1-\frac{1}{k+1})=\frac{1}{k+1}$, if $i=k+1$, then the probability is $\frac{1}{k+1}$ because it is only when we swap the latest $k+1$-th element with the given element, so for a given element to stay in $1\leq i \leq k+1$ position, the probability is thus $\frac{1}{k+1}$, which completes the proof.
		\end{proof} 
		Note, we can prove/explain the algorithm in another aspect. For a $k$ distinctive elements ordered set, there is $A_{k}^{k}$ possible combination of the sets and for a given element $\alpha$ be fixed in one position, we can generate $A_{k-1}^{k-1}$ sets, that is for a given element $\alpha$ in a fixed position from 1 to $k$, the probability is $\frac{A_{k-1}^{k-1}}{A_{k}^{k}}$, and for the $k+1$ rounds of the routine, the $\alpha$ to stay in the original position is thus we swap elements other than $\alpha$, that is 
		\begin{flalign}\label{prob_key_eq}
			\frac{A_{k-1}^{k-1}}{A_{k}^{k}} (1-\frac{1}{k+1})=\frac{(k-1)!}{k!}\frac{k}{k+1}=\frac{1}{k+1}
		\end{flalign}
		(\ref{prob_key_eq}) is what we want.
		
	
\end{document}